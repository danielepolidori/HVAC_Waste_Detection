\documentclass[conference]{IEEEtran}

\usepackage{cite}
\usepackage{amsmath,amssymb,amsfonts}
\usepackage{algorithmic}
\usepackage{graphicx}
\usepackage{textcomp}
\usepackage{xcolor}


\begin{document}

\title{Project report - Internet of things}

\author{\IEEEauthorblockN{Daniele Polidori}
\IEEEauthorblockA{\textit{University of Bologna}\\
daniele.polidori2@studio.unibo.it}}

\maketitle


\section{Introduction}
In this project I implemented an IoT system that monitors the temperature of an house, to prevent useless HVAC consumption, detecting it by rapid temperature changes. In this way, you can reduce energy usage, save money on the bills and is also a good procedure to follow for environmental reasons.\\
In this report I show the structure of the system that I've realized. Then I show the experimental setup and results that I've obtained. At the end I present some thoughts on the experiments' results.


\section{Project’s Architecture}
% iot pipeline
...

\section{Project’s Implementation}
% 1 file esp, 2 file py, influx, forecast, grafana
...

\section{Results}
% valori per l'eval (soglia varianza, tempi forecast, tempi queries)
% tabelle eval forecast e latency
%  considerazioni sui risultati
% sviluppi futuri
...

% \subsection{Maintaining the Integrity of the Specifications}
% ...

\end{document}
