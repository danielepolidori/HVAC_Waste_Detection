\documentclass[conference]{IEEEtran}

\usepackage{cite}
\usepackage{amsmath,amssymb,amsfonts}
\usepackage{algorithmic}
\usepackage{graphicx}
\usepackage{textcomp}
\usepackage{xcolor}


\begin{document}

\title{Project report - Internet of things}

\author{\IEEEauthorblockN{Daniele Polidori}
\IEEEauthorblockA{\textit{University of Bologna}\\
daniele.polidori2@studio.unibo.it}}

\maketitle


\section{Introduction}
I implemented an IoT system that monitors the temperature of an house, to prevent useless HVAC consumption, detecting it by rapid temperature changes. Reducing energy usage has good environmental reasons and, in this way, you can also save money on the bills.\\
In this report I show the components and the structure of the system that I've realized. Then I show the experimental setup and results that I've obtained; at the end, I present some thoughts on them.


\section{Project’s Architecture}
The system is composed by an ESP-WROOM-32 board linked to two DHT22 sensors (one placed inside the house and one outside) and to a LED.\\
The board periodically collects indoor and outdoor temperature values. They are constantly analysed: if the indoor temperature values rapidly change (going towards the values of outdoor ones) the LED is temporarily turned on, showing an alarm signal to the user. In this way, if the temperature change is caused by, e.g., an open window, you can save unnecessary HVAC waste. Based on all past data, the system makes a prediction of some future temperature values.\\
The temperature data are continuously sent to a gateway, that stores them, together with the alarm triggers and the forecasted values, on a local time-series database. All data are interactively visualized by a local web application, that shows them by means of charts.


\section{Project’s Implementation}
% 2 file py (librerie, coap req, mqtt pub, influx, forecast FBprophet)
% grafana

\subsection{ESP32 board}
I made the \texttt{data\_acquisition.ino} file to program the ESP32 board. I used the \texttt{Thing.CoAP} library to make the board act as a CoAP server and the \texttt{PubSubClient} library to make it act as a MQTT subscriber.\\
Through the CoAP protocol, the ESP32 is able to send the last temperature values collected, when asked.\\
Through the MQTT protocol, the board can receive some commands: to change the interval between consecutive sensor readings, to start or stop the sensor reading (in both cases you can decide for just one of the two sensors) and to turn on or off the LED. My laptop acts as a MQTT broker, through Mosquitto.

\subsection{Python scripts}
...


\section{Results}
% SETUP
% ogni quanto raccolgo temperature con esp, ogni quanto py le chiede (e le analizza)
% py fa partire i dht e sceglie il rate
% valori per l'eval (soglia varianza, tempi forecast, tempi queries)
% RISULTATI / EVALUATION (?)
% tabelle eval forecast e latency
%  considerazioni sui risultati
% considerazioni sul setup
% sviluppi futuri
...

% \subsection{...}
% ...

\end{document}
