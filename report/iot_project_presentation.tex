\documentclass{beamer}

\usepackage[super]{nth}


\title{HVAC Waste Detection - Project Presentation}
\author{Daniele Polidori}
\institute{\textit{Course of Internet of things}\\
\textit{University of Bologna}}
\date{Academic year 2023-24}

\setbeamertemplate{navigation symbols}{\tiny\insertframenumber}     % Inserisce il numero di pagina al posto dei simboli di navigazione
\setbeamercolor{navigation symbols}{fg=black}   % Colora di nero i simboli di navigazione


\begin{document}


{
\setbeamertemplate{footline}{}      % Elimina il footline in questa slide
\setbeamertemplate{navigation symbols}{}    % Lascia uno spazio vuoto al posto dei simboli di navigazione
\begin{frame}
 \titlepage     % Beamer's \maketitle
\end{frame}
}
%\addtocounter{framenumber}{-1}     % Non considera questa slide nel conteggio delle pagine


\begin{frame}{IoT system}

	It monitors the temperature of an house, to prevent useless HVAC consumption, detecting it by rapid temperature changes.
	
	\vfill

	\begin{block}

		\begin{columns}[onlytextwidth,T]
		
			\column{\dimexpr\linewidth-60mm-5mm}

			Composed by an \textbf{ESP-WROOM-32} board linked to:
			\begin{itemize}
				\item 1 \textbf{indoor DHT22},
				\item 1 \textbf{outdoor DHT22},
				\item 1 \textbf{LED}.
			\end{itemize}

			\column{60mm}
			\includegraphics[scale=0.04]{figures/figure_esp.jpg}

		\end{columns}
	\end{block}
\end{frame}


\begin{frame}{Data acquisition - ESP32}

	\begin{description}[\texttt{PubSubClient}]		% Indico la lunghezza della label piu' lunga, cosi' che siano tutte allineate a dx
		\item[\texttt{Thing.CoAP}]: To act as a CoAP server.
		\item[\texttt{PubSubClient}]: To act as a MQTT subscriber.
	\end{description}

	\vfill

	Through CoAP, the ESP32 is able to send the latest collected indoor and outdoor temperature value, when asked.

	\vfill

	Through MQTT, the board can receive some commands:
	\begin{itemize}
		\item to start or stop the sensors reading,
		\item to change their sampling rate,
		\item to turn on or off the LED.
	\end{itemize}
	My laptop acts as a MQTT broker, through Mosquitto.

\end{frame}


\begin{frame}{Data proxy - \nth{1} Python script}

	\begin{description}[\texttt{influxdb-client}]		% Indico la lunghezza della label piu' lunga, cosi' che siano tutte allineate a dx
		\item[\texttt{paho-mqtt}]: To act as a MQTT publisher.
		\item[\texttt{aiocoap}]: To act as a CoAP client.
		\item[\texttt{influxdb-client}]: To store data.
	\end{description}

	\vfill

	Initially, through MQTT, the application gives commands, to the ESP32, to start the sensors reading and to set their sampling rate.
	
	\vfill
	
	Periodically, through CoAP, the script requests, to the board, the latest collected indoor and outdoor temperature value.\\
	It continuously stores these values on a local InfluxDB instance.
	
	\vfill
	
	The network latency, between the temperature value request and its reception, is continuously monitored and, after a while from the beginning, the mean value is calculated.

\end{frame}


\begin{frame}{Data analytics - \nth{2} Python script (1/2)}

	\begin{description}[\texttt{influxdb-client}]		% Indico la lunghezza della label piu' lunga, cosi' che siano tutte allineate a dx
		\item[\texttt{influxdb-client}]: To get and store data.
		\item[\texttt{prophet}]: To forecast future temperature values.
		\item[\texttt{paho-mqtt}]: To act as a MQTT publisher.
	\end{description}

	\vfill

	At the beginning, the script gets all past temperature values to forecast some indoor and outdoor values.\\As the times are reached, the application stores the predicted values on the database.
	
	\vfill
	
	Ciclically, the application retrieves some of the latest temperature values and analyses them to check a possible HVAC waste.\\
	When the alarm goes off, the script stores the event on the database and, through MQTT, gives the command, to the ESP32, to turn on the LED; when the risk has passed, the script gives the command to turn it off.

\end{frame}


\begin{frame}{Data analytics - \nth{2} Python script (2/2)}

	The alarm goes off if:
	\begin{itemize}
		\item the indoor temperature is changing rapidly \eqref{eq_varianza} and
		\item the indoor temperature is approaching the outdoor one \eqref{eq_media}.
	\end{itemize}
	
	\vfill
	
	Mathematically speaking:
	\begin{equation}
	var(i_1, i_2, \dots, i_n) > threshold \label{eq_varianza}
	\end{equation}
	\begin{equation}
	min(i_n, o_n) < mean(i_1, i_2, \dots, i_n) < max(i_n, o_n) \label{eq_media}
	\end{equation}
	where $i$ is the indoor temperature, $o$ is the outdoor temperature and $t_1, t_2, \dots, t_n$ are the $n$ latest temperature values retreived: $t_1$ is the newest and $t_n$ is the farthest.

\end{frame}


\begin{frame}{Data visualization}

 	Local Grafana instance that shows:
	\begin{block}

		\begin{columns}[onlytextwidth,T]
		
			\column{\dimexpr\linewidth-65mm-5mm}

			\begin{itemize}
				\item the collected temperature values,
				\item the forecasted ones,
				\item the counting of the alarm events.
			\end{itemize}

			\column{70mm}
			\includegraphics[scale=0.10]{figures/figure_grafana.png}

		\end{columns}
	\end{block}
\end{frame}


\begin{frame}{Setup (data proxy)}

	ESP32)
	\begin{description}[Mean network latency evaluation]
		\item[Indoor DHT sampling rate]: 3 sec.
		\item[Outdoor DHT sampling rate]: 20 sec.
	\end{description}
	
	\vfill
	
	Data acquisition process)
	\begin{description}[Mean network latency evaluation]
		\item[Latest temperatures request]: every 5 sec.
		\item[Mean network latency evaluation]: after 1 hour.
	\end{description}

\end{frame}


\begin{frame}{Setup (data analytics)}

	Forecast)
	\begin{description}[Num. of values retreived]
		\item[Data obtained]: on start.
		\item[Temperatures collection]: past month (unevenly).
		\item[aggregateWindow]: every 20 sec (mean function).
		\item[Num. of values retreived]: 6500 indoor, 6500 outdoor.
		\item[Forecast]: every 10 min, for 6 times.
	\end{description}
	
	\vfill
	
	Alarm)
	\begin{description}[Num. of values retreived]
		\item[Data obtained]: every 30 sec.
		\item[Temperatures collection]: latest 2 min.
		\item[Alarm threshold \eqref{eq_varianza}]: $0.03$.
	\end{description}

\end{frame}


\end{document}
